\documentclass[journal,12pt,twocolumn]{IEEEtran}
\usepackage{setspace}
\usepackage{gensymb}
\singlespacing
\usepackage[cmex10]{amsmath}
\usepackage{amsthm}
\usepackage{mathrsfs}
\usepackage{txfonts}
\usepackage{stfloats}
\usepackage{bm}
\usepackage{cite}
\usepackage{cases}
\usepackage{subfig}
\usepackage{longtable}
\usepackage{multirow}
\usepackage{enumitem}
\usepackage{mathtools}
\usepackage{tikz}
\usepackage{circuitikz}
\usepackage{verbatim}
\usepackage[breaklinks=true]{hyperref}
\usepackage{tkz-euclide} % loads  TikZ and tkz-base
\usepackage{listings}
\usepackage{color}    
\usepackage{array}    
\usepackage{longtable}
\usepackage{calc}     
\usepackage{multirow} 
\usepackage{hhline}   
\usepackage{ifthen}   
\usepackage{lscape}     
\usepackage{chngcntr}
\DeclareMathOperator*{\Res}{Res}
\renewcommand\thesection{\arabic{section}}
\renewcommand\thesubsection{\thesection.\arabic{subsection}}
\renewcommand\thesubsubsection{\thesubsection.\arabic{subsubsection}}

\renewcommand\thesectiondis{\arabic{section}}
\renewcommand\thesubsectiondis{\thesectiondis.\arabic{subsection}}
\renewcommand\thesubsubsectiondis{\thesubsectiondis.\arabic{subsubsection}}
\renewcommand\thetable{\arabic{table}}
% correct bad hyphenation here
\hyphenation{op-tical net-works semi-conduc-tor}
\def\inputGnumericTable{}                                 %%

\lstset{
%language=C,
frame=single, 
breaklines=true,
columns=fullflexible,
literate=
    {->}{$\rightarrow{}$}{1}
    {~}{$\sim$}{1},
}
%\lstset{
%language=tex,
%frame=single, 
%breaklines=true
%}

\begin{document}
\newtheorem{theorem}{Theorem}[section]
\newtheorem{problem}{Problem}
\newtheorem{proposition}{Proposition}[section]
\newtheorem{lemma}{Lemma}[section]
\newtheorem{corollary}[theorem]{Corollary}
\newtheorem{example}{Example}[section]
\newtheorem{definition}[problem]{Definition}
\newcommand{\BEQA}{\begin{eqnarray}}
\newcommand{\EEQA}{\end{eqnarray}}
\newcommand{\define}{\stackrel{\triangle}{=}}
\bibliographystyle{IEEEtran}
\providecommand{\figcap}[3]{
    \begin{figure}[!ht]
        \centering
        \includegraphics[width=\columnwidth]{#1}
        \caption{#2}
        \label{fig:#3}
    \end{figure}
}
\providecommand{\mbf}{\mathbf}
\providecommand{\pr}[1]{\ensuremath{\Pr\left(#1\right)}}
\providecommand{\qfunc}[1]{\ensuremath{Q\left(#1\right)}}
\providecommand{\sbrak}[1]{\ensuremath{{}\left[#1\right]}}
\providecommand{\lsbrak}[1]{\ensuremath{{}\left[#1\right.}}
\providecommand{\rsbrak}[1]{\ensuremath{{}\left.#1\right]}}
\providecommand{\brak}[1]{\ensuremath{\left(#1\right)}}
\providecommand{\lbrak}[1]{\ensuremath{\left(#1\right.}}
\providecommand{\rbrak}[1]{\ensuremath{\left.#1\right)}}
\providecommand{\cbrak}[1]{\ensuremath{\left\{#1\right\}}}
\providecommand{\lcbrak}[1]{\ensuremath{\left\{#1\right.}}
\providecommand{\rcbrak}[1]{\ensuremath{\left.#1\right\}}}
\theoremstyle{remark}
\newtheorem{rem}{Remark}
\newcommand{\sgn}{\mathop{\mathrm{sgn}}}
\providecommand{\abs}[1]{\left\vert#1\right\vert}
\providecommand{\res}[1]{\Res\displaylimits_{#1}} 
\providecommand{\norm}[1]{\left\lVert#1\right\rVert}
\providecommand{\mtx}[1]{\mathbf{#1}}
\providecommand{\mean}[1]{E\left[ #1 \right]}
\providecommand{\fourier}{\overset{\mathcal{F}}{ \rightleftharpoons}}
\providecommand{\system}[1]{\overset{\mathcal{#1}}{ \longleftrightarrow}}
\newcommand{\solution}{\noindent \textbf{Solution: }}
\newcommand{\cosec}{\,\text{cosec}\,}
\providecommand{\dec}[2]{\ensuremath{\overset{#1}{\underset{#2}{\gtrless}}}}
\newcommand{\myvec}[1]{\ensuremath{\begin{pmatrix}#1\end{pmatrix}}}
\newcommand{\mydet}[1]{\ensuremath{\begin{vmatrix}#1\end{vmatrix}}}
\let\vec\mathbf
\def\putbox#1#2#3{\makebox[0in][l]{\makebox[#1][l]{}\raisebox{\baselineskip}[0in][0in]{\raisebox{#2}[0in][0in]{#3}}}}
     \def\rightbox#1{\makebox[0in][r]{#1}}
     \def\centbox#1{\makebox[0in]{#1}}
     \def\topbox#1{\raisebox{-\baselineskip}[0in][0in]{#1}}
     \def\midbox#1{\raisebox{-0.5\baselineskip}[0in][0in]{#1}}

\vspace{3cm}
\title{Installation of De-Googled Android on Smartphones}
\author{Gautam Singh}
\maketitle
\tableofcontents
\bigskip

\begin{abstract}
    This document contains an illustration of the process to install
    De-Googled Android-based operating systems.
\end{abstract}

\section{Important Disclaimer}

\begin{enumerate}[label=\thesection.\arabic*
,ref=\thesection.\theenumi]
\item The steps presented in this manual may vary for different phones and 
OEMs. Here, we have considered a \textbf{OnePlus 6} (\textit{enchilada}) and
presented the corresponding steps.
\item This guide is \textbf{not} a substitute for any official documentation
from the OEM of the phone or the communities behind the custom ROMs. Please
consult the official installation guides from the respective websites.
\item The author(s) of this document are \textbf{not} responsible for bricked 
devices, dead SD cards, thermonuclear war, or you getting fired because the 
alarm app failed.
\item Your warranty will be void if you tamper with any part of your device/ 
software.
\item This document deals with the installation of LineageOS and /e/OS on
phones that ship with stock Android ROMs.
\end{enumerate}

\section{Requirements}

\begin{enumerate}[label=\thesection.\arabic*
,ref=\thesection.\theenumi]
\item An Android phone.
\item A laptop/desktop, preferably running Windows or MacOS. Linux may work, 
but not always, since drivers released by the OEM are usually not
compatible with Linux.
\item A USB C to USB cable (essentially the charging cable) of your
phone.
\item (Optional) USB 2.0/3.0 hub. Some phones may work only on older USB
ports, so this might be handy.
\end{enumerate}

\section{Setting Up}

\begin{enumerate}[label=\thesection.\arabic*
,ref=\thesection.\theenumi]
\item Download the appropriate drivers for your phone from the website of the
OEM.
\item Download the latest Android SDK platform tools
\begin{lstlisting}
$ wget https://dl.google.com/android/repository/platform-tools-latest-darwin.zip -O ~/Downloads/platform-tools.zip
\end{lstlisting}
Unzip the downloaded \texttt{platform-tools.zip} file, which contains the 
fastboot and adb binaries.
\item Download the script to ensure consistency of paritions
\begin{lstlisting}
$ wget https://mirrorbits.lineageos.org/tools/copy-partitions-20220613-signed.zip
\end{lstlisting}
\item Download the recovery and rootfs .img files to flash onto your phone,
as shown:
\begin{enumerate}
    \item LineageOS
    \begin{lstlisting}
$ wget https://mirrorbits.lineageos.org/recovery/enchilada/20230124/lineage-20.0-20230124-recovery-enchilada.img
$ wget https://mirrorbits.lineageos.org/full/enchilada/20230124/lineage-20.0-20230124-nightly-enchilada-signed.zip
    \end{lstlisting}
    \item /e/ OS
    \begin{lstlisting}
$ wget https://images.ecloud.global/dev/enchilada/recovery-e-1.7-s-20230111250406-dev-enchilada.img
$ wget https://images.ecloud.global/dev/enchilada/e-1.7-s-20230111250406-dev-enchilada.zip
    \end{lstlisting}
\end{enumerate}
\item Finally, do not forget to take a backup of all your data on the phone 
(if needed).
\end{enumerate}

\section{Unlocking the Bootloader}

\textbf{Note:} If you have already unlocked the bootloader, please skip this
section.
\begin{enumerate}[label=\thesection.\arabic*
,ref=\thesection.\theenumi]
\item In your Android phone, go to 
\begin{lstlisting}
Settings -> About Phone.
\end{lstlisting}
\item Tap on Build Number 7 times (or as many times as required by the phone)
to enable developer mode. See Fig. \ref{fig:build-no}
\figcap{figs/build-number.jpg}{Build number of the phone in Android.}{build-no}
\item Navigate to 
\begin{lstlisting}
Settings -> Developer Options
\end{lstlisting} 
and toggle the following options:
\begin{enumerate}
    \item OEM Unlocking (Fig. \ref{fig:oem})
    \item USB Debugging (Fig. \ref{fig:adb})
\end{enumerate}
\figcap{figs/devops1.jpg}{OEM unlocking in developer options.}{oem}
\figcap{figs/devops2.jpg}{USB debugging in developer options.}{adb}
\item Plug in the phone to your laptop and open a terminal window.
\item Run the following commands and approve USB debugging when the pop-up
appears on the phone, as in Fig. \ref{fig:usb-debugging}
\begin{lstlisting}
$ adb kill-server
$ adb reboot bootloader
\end{lstlisting}
\item The phone will then reboot to its bootloader, as shown in Fig. 
\ref{fig:bootloader}. To verify this, in the same terminal, run
\begin{lstlisting}
$ fastboot devices
\end{lstlisting}
A serial number corresponding to your device should appear.
\figcap{figs/usb-debugging.jpg}{Pop-up dialog to approve USB debugging from a system.}{usb-debugging}
\item To unlock the bootloader, run
\begin{lstlisting}
$ fastboot oem unlock
\end{lstlisting}
and navigate to the option to unlock the bootloader using volume keys, as in
Fig. \ref{fig:unlock-bootloader}. To enter your choice, press the power button.
\figcap{figs/bootloader.jpg}{The bootloader of the OnePlus 6 (enchilada).}{bootloader}
\figcap{figs/unlock.jpg}{Unlocking the bootloader of the OnePlus 6 (enchilada).}{unlock-bootloader}
\item The phone will reboot into Android after performing a full wipe of
the data. Enable USB debugging as before and reboot into the bootloader.
\end{enumerate}

\section{Flashing and Booting Custom ROMs}

\begin{enumerate}[label=\thesection.\arabic*
,ref=\thesection.\theenumi]
\item Flash the recovery file to the boot partition.
\begin{lstlisting}
$ fastboot flash boot /path/to/recovery.img
\end{lstlisting}
\item Reboot to recovery mode, using any one of the following (see Fig. 
\ref{fig:recovery-fastboot}).
\begin{enumerate}
\item From fastboot, either using
\begin{lstlisting}
$ fastboot reboot recovery
\end{lstlisting}
or navigating to \texttt{Recovery Mode} using the volume and power keys.
\item By long pressing the power and volume up button when the device
powered off.
\end{enumerate}
\figcap{figs/recovery.jpg}{Toggle to recovery mode in fastboot.}{recovery-fastboot}
\item \label{item:sideload}
Once booted to recovery, as in either Fig. \ref{fig:los-recovery} or Fig. 
\ref{fig:eos-recovery}, navigate to 
\begin{lstlisting}
Apply Update -> Apply From ADB -> ADB Sideload
\end{lstlisting}
From the computer terminal, type
\begin{lstlisting}
$ adb sideload /path/to/copy-partitions.zip
\end{lstlisting}
When the file is uploaded, exit status 0 displays on the phone, and on the
computer \texttt{Total xfer: 1.00x} will appear on the terminal.
\figcap{figs/eos-recovery.jpg}{/e/OS recovery interface.}{eos-recovery}
\figcap{figs/los-recovery.jpg}{LineageOS recovery interface.}{los-recovery}
\item Go to recovery, and navigate to
\begin{lstlisting}
Advanced -> Reboot to Recovery
\end{lstlisting}
to reboot to recovery again.
\item Once back in recovery, sideload the rom zip file as in step 
\ref{item:sideload}. Optionally, you may choose to flash a root patch such
as SuperSU or Magisk to root your phone at this point.
\item When the sideload is done, go back to the main menu, and press
\texttt{Reboot System Now} to reboot to the new OS. See Fig. \ref{fig:eos} and
Fig. \ref{fig:los}.
\figcap{figs/eos.png}{/e/OS interface.}{eos}
\figcap{figs/los.png}{LineageOS interface.}{los}
\end{enumerate}
\end{document}
